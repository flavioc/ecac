\documentclass[a4paper]{llncs}
\usepackage{graphicx}
\usepackage{url}
\usepackage{longtable}
\usepackage[portuguese]{babel}
\usepackage[latin1]{inputenc}
\urldef{\mails}\path|{ei05011,ei05028}@fe.up.pt|
\bibliographystyle{splncs}
\begin{document}
\title{Descri��o do dataset ``Adult Database''}
\author{Fl�vio Cruz\inst{1} \and Jo�o Azevedo\inst{1}}
\institute{Faculdade de Engenharia da Universidade do Porto\\
    Rua Dr. Roberto Frias, s/n 4200-465 Porto PORTUGAL
    \mails}
\maketitle

\begin{abstract}
Usado de forma extremamente frequente em testes experimentais relacionados com
data mining, a ``Adult Database'' reveste-se de algumas propriedades
interessantes que a tornam bastante �til para o teste da efic�cia de algoritmos
de classifica��o. Este relat�rio pretende descrever a ``Adult Database'' de um
ponto de vista estat�stico e tendo em conta o seu suporte para tarefas de data
mining.
\end{abstract}

\section{Introdu��o}
\label{sec:intro}

\section{Formato dos Dados}

O ``Adult Database'' consiste de 48842 entradas de informa��o de pessoas
recolhida durante um censos de 1994 realizado no estado da Calif�rnia, nos
Estados Unidos da Am�rica. Dos dados globais, foi extra�da uma amostra
representativa de pessoas com idades entre os 16 e os 100 anos e um peso
superior a 1.

\section{An�lise dos Dados}

\section{Trabalho Futuro}

\section{Conclus�es}

\begin{thebibliography}{1}
\end{thebibliography}

\end{document}
